% Вказуємо клас документа, розмір шрифту та формат паперу
\documentclass[14pt,a4paper,twoside]{article}

% --- Шрифти ---
\usepackage{lmodern} % Сучасні шрифти для кращої сумісності та масштабування

% --- Кодування та Локалізація ---
\usepackage[utf8]{inputenc} % Встановлює кодування вводу як UTF-8
\usepackage[T2A]{fontenc}   % Встановлює кодування шрифту для кирилиці
\usepackage[english,ukrainian]{babel} % Підтримка англійської та української мов

% --- Графіка ---
\usepackage{graphicx} % Для вставки графічних зображень (jpg, png, etc.)

% --- Геометрія сторінки ---
\usepackage[left=25.4mm, right=12.7mm, top=20mm, bottom=20mm]{geometry} % Встановлює поля документа

% --- Відступи та форматування тексту ---
\usepackage{indentfirst} % Додає відступ для першого абзацу після заголовка розділу

% --- Математичні пакети ---
\usepackage{amsmath}  % Додаткові математичні символи та формули
\usepackage{amsfonts} % Додаткові математичні шрифти (наприклад, \mathbb)
\usepackage{amssymb}  % Додаткові математичні символи (наприклад, \nexists, \varnothing)

% --- Таблиці ---
\usepackage{array}    % Для додаткових опцій форматування таблиць

% --- Форматування заголовків ---
\usepackage{titlesec} % Для кастомізації формату заголовків (розділів, підрозділів, тощо)

% --- Заголовки та підвали сторінок ---
\usepackage{fancyhdr} % Для кастомізації заголовків та підвалів сторінок

% --- Міжрядкові інтервали ---
\usepackage{setspace} % Для керування міжрядковими інтервалами

% --- Додаткові утиліти (Lorem Ipsum, посилання на розділи, тощо) ---
\usepackage{lipsum}   % Для генерації "заповнювача тексту" (Lorem Ipsum)
\usepackage{nameref}  % Для створення посилань на назви розділів у тексті

% --- Гіперпосилання (зазвичай підключається останнім) ---
\usepackage{hyperref} % Для створення активних гіперпосилань у PDF

% --- Налаштування формату заголовків та інших елементів ---
\titleformat{\section}[block]{\normalfont\Large\bfseries}{\thesection}{1em}{}
\titleformat{\subsection}[block]{\normalfont\large\bfseries}{\thesubsection}{1em}{}
\titleformat{\subsubsection}[block]{\normalfont\normalsize\bfseries}{\thesubsubsection}{1em}{}

% --- Відступи після заголовків ---
\titlespacing*{\paragraph}{0pt}{\baselineskip}{1em}
\titlespacing*{\subparagraph}{0pt}{\baselineskip}{1em}

% --- Налаштування сторінок ---
\pagestyle{fancy}
\fancyhf{}
\fancyhead[RO]{\thepage}
\fancyhead[LE]{\thepage}
\fancyhead[CE,CO]{}
\fancypagestyle{plain}{
	\fancyhf{}
}

\begin{document}
	\onehalfspacing  % Встановлення інтервалу 1.5
	\title{Лабораторна робота: Створення наукової статті в LaTeX з клікабельними посиланнями}
	\author{Dr. Sergii Burian \and Vyacheslav Yevdokimov}
	\date{\today}
	\maketitle
	\thispagestyle{plain}
	
	\newpage  % Перехід на нову сторінку
	\thispagestyle{empty}  % Зробити цю сторінку порожньою (без номерів сторінок та заголовків)
	\mbox{}  % Додати порожній "блок", щоб LaTeX вважав сторінку "зайнятою"
	
	\newpage  % Перехід на нову сторінку
	
	\tableofcontents  % Генерація змісту
	
	\newpage  % Перехід на нову сторінку
	
	\section*{Анотація}
	Цей документ служить як приклад та завдання для студентів, які вивчають LaTeX. Завдання полягає в створенні наукової статті з використанням різних типів форматування, вставки математичних формул, створення таблиць, списків та клікабельних посилань. Детальні приклади представлені у відповідних розділах цього документа.
	
	\newpage
	
	\section{Вступ}
	Команда \texttt{\textbackslash lipsum[1]} використовується для генерації "заповнювача тексту" в LaTeX документах. Це корисно, коли вам потрібно швидко додати текст в документ для демонстрації форматування, структури або інших аспектів документа, не звертаючи уваги на зміст.
	
	Пакет lipsum надає доступ до рядків "Lorem Ipsum", які часто використовуються як заповнювач тексту в різних дизайнерських і видавничих задачах. Команда \texttt{\textbackslash lipsum[1]} вставляє перший абзац "Lorem Ipsum" тексту. Ви також можете вказати діапазон абзаців, наприклад, \texttt{\textbackslash lipsum[1-5]} вставить перші п'ять абзаців. Наступний абзац є прикладом застосування  \texttt{\textbackslash lipsum[5]}.
	
	\lipsum[4]
	
	\section{Пояснення елементів преамбули}
	
	\begin{itemize}
		\item \textbf{\textbackslash documentclass[14pt,a4paper,twoside]\{article\}}: Вказує клас документа (article), глобальний розмір шрифту (14pt), розмір паперу (a4paper), і двосторонню друковану версію.
		
		\item \textbf{\textbackslash usepackage\{lmodern\}}: Сучасні шрифти для кращої сумісності та масштабування.
		
		\item \textbf{\textbackslash usepackage[utf8]\{inputenc\}}: Встановлює кодування вводу як UTF-8.
		
		\item \textbf{\textbackslash usepackage[T2A]\{fontenc\}}: Встановлює кодування шрифту для кирилиці.
		
		\item \textbf{\textbackslash usepackage[english,ukrainian]\{babel\}}: Підтримка англійської та української мов.
		
		\item \textbf{\textbackslash usepackage\{graphicx\}}: Для вставки графічних зображень.
		
		\item \textbf{\textbackslash usepackage[left=25.4mm, right=12.7mm, top=20mm, bottom=20mm]\{geometry\}}: Встановлює поля документа.
		
		\item \textbf{\textbackslash usepackage\{indentfirst\}}: Додає відступ для першого абзацу після заголовка розділу.
		
		\item \textbf{\textbackslash usepackage\{amsmath, amsfonts, amssymb\}}: Додаткові математичні символи, шрифти та формули.
		
		\item \textbf{\textbackslash usepackage\{array\}}: Для додаткових опцій форматування таблиць.
		
		\item \textbf{\textbackslash numberwithin\{equation\}\{section\}}: Додає номер розділу до номерів формул.
		
		\item \textbf{\textbackslash numberwithin\{table\}\{section\}}: Додає номер розділу до номерів таблиць.
		
		\item \textbf{\textbackslash usepackage\{titlesec\}}: Для кастомізації формату заголовків.
		
		\item \textbf{\textbackslash usepackage\{fancyhdr\}}: Для кастомізації заголовків та підвалів сторінок.
		
		\item \textbf{\textbackslash usepackage\{setspace\}}: Для керування міжрядковими інтервалами.
		
		\item \textbf{\textbackslash usepackage\{lipsum\}}: Для генерації "заповнювача тексту" (Lorem Ipsum).
		
		\item \textbf{\textbackslash usepackage\{nameref\}}: Для створення посилань на назви розділів у тексті.
		
		\item \textbf{\textbackslash usepackage\{hyperref\}}: Для створення активних гіперпосилань у PDF.
		
		\item \textbf{\textbackslash fancyhead[RO]\{\textbackslash thepage\}}: Нумерація для непарних сторінок справа.
		
		\item \textbf{\textbackslash fancyhead[LE]\{\textbackslash thepage\}}: Нумерація для парних сторінок зліва.
		
		\item \textbf{\textbackslash titlespacing*\{\textbackslash paragraph\}\{...\}}: Відступи після заголовків параграфів.
	\end{itemize}
	
	
	\subsection{Додаткові вимоги (Оцінюється в 4 бали)}
	\begin{itemize}
		\item Використовуйте двосторонній друк (\texttt{twoside}) та формат паперу A4.
		\item Використати титульну сторінку та додати порожню сторінку після титульної і теж її не нумерувати.
		\item Встановіть поля документа: ліве – 2.54 см, праве – 1.27 см, верхнє і нижнє – 2.0 см.
		\item Використовуйте кастомізовані заголовки та колонтитули для сторінок: нумерація для непарних сторінок справа, нумерація для парних сторінок зліва, порожньо для центральної частини верхнього колонтитула на всіх сторінках
		\item Всі математичні формули та таблиці повинні бути правильно вирівняні та позначені.
		\item Всі посилання на формули, таблиці та інші елементи мають бути клікабельними. Для цього використовуйте пакет \texttt{hyperref}.
	\end{itemize}
	
	\subsection{Використання формул (Оцінюється в 3 бали)}
	
	
	\subsubsection{Більш детальна інструкція до використання формул}
	
	У підрозділі додайте приклади простих та складних математичних формул. Використовуйте різні математичні оточення (\texttt{equation}, \texttt{multline}, \texttt{align} тощо). Позначте кожну формулу за допомогою \texttt{\textbackslash label} і вставте посилання на неї в тексті за допомогою \texttt{\textbackslash ref}.
	
	У наступному підрозділі наведено приклад того, як Ви повинні продемонструвати свої навички використання LaTeX при записі формул в рядку, окремим рядком та записавши формулу в кілька рядків. Для цього кожному з Вас дається індивідуальне завдання, яке наводиться в таблиці~\ref{tab:taylor_series} у  \hyperref[sec:appendix1]{Додатку А}.
	
	Проробіть усі аналогічні дії, але зі своїм завданням!
	
	\subsubsection{Написання складних формул}
	
	\paragraph{Математичні формули.}
	
	Розглянемо функцію  \( \cot(x)\) та її розклад у ряд Маклорена. Для функції \( \cot(x)\), її розклад у ряд Тейлора можна записати наступним чином:
	\begin{equation} \label{eq:cot_general}
		 \cot(x) = \sum_{n=1}^{\infty} \frac{(-1)^{n+1} x^{2n-1}}{2n-1} 
	\end{equation}
	Позначимо \( n \)-ий член цього ряду як \( q_n \):
	\begin{equation} \label{eq:q_sum_n}
		q_n = \frac{(-1)^{n+1} x^{2n-1}}{2n-1}
	\end{equation}
	
	\paragraph{Складні математичні формули.}
	
	Тепер розглянемо суму ряду \( \cot(x)  = q_0 + q_1 + q_2 + \ldots\) в контексті виразу \eqref{eq:q_sum_n}. Для цього можна використати команди \text{align} та \text{multline}:
	\begin{align} \label{eq:cot_sum}
		& \cot(x) = q_0 + q_1 + q_2 + \ldots  = \\ 
		&= {x^{-1}} - \frac{1}{3}x - \frac{1}{45}x^3 - \frac{2}{945}x^5 - \ldots \nonumber
	\end{align}
	\begin{multline} \label{eq:sum_series_10_terms}
		\cot(x) ={x^{-1}} - \frac{1}{3}x - \frac{1}{45}x^3 - \frac{2}{945}x^5 -  \\
		- \frac{1}{4725}x^7 - \frac{2}{93555}x^9 - \frac{1382}{638512875}x^{11} - \frac{4}{18243225}x^{13} - \\
	    - \frac{3617}{162820783125}x^{15} - \frac{87734}{38979295480125}x^{17} - \ldots
	\end{multline}
	
	Якщо у нас є відомий \( n \)-ий член ряду \( q_n \), то ми можемо знайти \( (n+1) \)-ий член \( q_{n+1} \) через нього. Для цього нам потрібно знайти рекурентний множник \( R \): 
	\begin{equation} \label{eq:R_def}
		R = \frac{q_{n+1}}{q_n}.
	\end{equation}
	
	Вираз для \( q_{n+1} \) можна записати наступним чином:
	\begin{equation} \label{eq:qn1_def}
		q_{n+1} = \frac{(-1)^{n+2} x^{2n+1}}{2n+1}
	\end{equation}
	
	Підставляючи \eqref{eq:qn1_def} та \eqref{eq:q_sum_n} у \eqref{eq:R_def}, отримуємо:
	\begin{equation} \label{eq:R_simplified}
		R = -\frac{x^{2n+1} (2n-1)}{x^{2n-1} (2n+1)}
	\end{equation}
	
	Тепер, знаючи \( R \) і \( q_n \), можна знайти \( q_{n+1} \) як \( q_{n+1} = R \cdot q_n \).
	
	Для детального розгляду, дивіться формули \eqref{eq:cot_general}, \eqref{eq:q_sum_n}, \eqref{eq:R_def}, \eqref{eq:qn1_def}, і \eqref{eq:R_simplified}.
	
	
	% Таблиця
	\subsection{Використання таблиць (Оцінюється в 3 бали)}
	
	\subsubsection{Більш детальна інструкція до використання таблиць}
	
	Створіть таблицю з декількома рядками та стовпцями. Використовуйте \texttt{\textbackslash caption} для додавання заголовка таблиці та \texttt{\textbackslash label} для можливості посилання на неї. Вставте посилання на таблицю в тексті за допомогою \texttt{\textbackslash ref}
	
	Ось приклад таблиці:
	
	
	\begin{table}[h]
		\centering
		\caption{Таблиця істинності для 20-го завдання} \label{sec:tabl_6}
		\label{tab:truth_table_20}
		\begin{tabular}{|c|c||c|c||c|}
			\hline
			\( A \) & \( B \) & \( \lnot A \lor B \) & \( \lnot B \) & \( \lnot B \land (\lnot A \lor B)  \) \\
			\hline
			\hline
			1 & 1 & 1 & 0 & 0 \\
			\hline
			1 & 0 & 0 & 1 & 0 \\
			\hline
			0 & 1 & 1 & 0 & 0\\
			\hline
			0 & 0 & 1 & 1 & 1 \\
			\hline
		\end{tabular}
	\end{table}
	
	
	
	Кожен знайде своє завдання для побудови таблиці істинності в таблиці~\ref{tab:individual_tasks} з індивідуальними завданнями у \hyperref[sec:appendix2]{Додатку Б}. Так, у мене це завдання під №6 наведено у  \hyperref[sec:tabl_6] {таблиці 1}.
	
	
	\section{Висновки}
	
	У висновку коротко опишіть, що було найскладніше для виконання в цій лабораторній роботі. 
	Також порекомендуйте книгу з LaTeX, яку Ви читали, або про яку чули і зробіть на неї посилання в списку літератури за зразком, що нижче.
	
	Найскладніше для виконання в цій лабораторній роботі було зрозуміти, які символи є обовязковими для використання, а які можна не писати. Також незрозумілим залишається: чому використання апострофу з української розкладки у слові видає помилку, хоча підтримка українськї мови є (тоді як з використанням апострофу з англійської розкладки проблем не виникає), також незрозуміло як при підклченоу пакеті \textbf{lmodern} не працює команда \textit{\textbackslash textbf та \textbackslash textit} навіть на англійських шрифтах.
	
	Для тих, хто тільки починає вивчати LaTeX, книга \textit{"LaTeX for beginners"} автора Hare Jean є відмінним вступом в цю тему~\cite{latexbegin}. Окрім того опис основних команд можна знайти на сайті \href{https://www.overleaf.com/learn}{overleaf.com}.
	
	Якщо ви вже маєте певний досвід роботи з LaTeX, але хочете поглибити свої знання, рекомендую звернутися до книги \textit{"Intermediate Latex"}~\cite{latexintermid}, яка є вичерпним джерелом інформації з цієї теми.\cite{ozdemir2023quick}
	
	Для вивчення специфічних аспектів LaTeX, таких як математичне форматування або створення графіки, книга \textit{"The LaTeX Companion"} може надати цінні поради та приклади~\cite{latexguide}.
	
	\newpage
	
	\bibliographystyle{plain}
	\bibliography{lib}
	\newpage
	\appendix
	
	\section{Перший додаток}\label{sec:appendix1}
	\begin{table}[h!]
		\centering
		\caption{Варіанти індивідуальних завдань для використання формул}
		\label{tab:taylor_series}
		\begin{tabular}{|c|c||c||c|}
			\hline
			№ & Функція & Розклад в ряд Тейлора & Інтервал збіжності \\
			\hline
			\hline
			1 & \( \ln(1+x) \) & \( \displaystyle\sum_{n=1}^{\infty} \frac{(-1)^{n+1} x^n}{n} \) & \( -1 < x \leq 1 \) \\
			\hline
			2 & \( \sqrt{1+x} \) & \( \displaystyle\sum_{n=0}^{\infty} \frac{(-1)^n (2n)! x^n}{2^{2n} (n!)^2 (2n+1)} \) & \( |x| \leq 1 \) \\
			\hline
			3 & \( \frac{1}{\sqrt{1+x}} \) & \( \displaystyle\sum_{n=0}^{\infty} \frac{(-1)^n (2n)! x^n}{2^{2n} (n!)^2 (2n+1)} \) & \( -1 < x \leq 1 \) \\
			\hline
			4 & \( \cos(x) \) & \( \displaystyle\sum_{n=0}^{\infty} \frac{(-1)^n x^{2n}}{(2n)!} \) & \( x \in \mathbb{C} \) \\
			\hline
			5 & \( \tan(x) \) & \( \displaystyle\sum_{n=1}^{\infty} \frac{(-1)^{n+1} x^{2n-1}}{(2n-1)} \) & \( |x| < \frac{\pi}{2} \) \\
			\hline
			6 & \( \cot(x) \) & \( \displaystyle\sum_{n=1}^{\infty} \frac{(-1)^{n+1} x^{2n-1}}{(2n-1)} \) & \( 0 < |x| < \pi \) \\
			\hline
			7 & \( \sec(x) \) & \( \displaystyle\sum_{n=0}^{\infty} \frac{(-1)^n x^{2n}}{(2n)!} \) & \( |x| < \frac{\pi}{2} \) \\
			\hline
			8 & \( \csc(x) \) & \( \displaystyle\sum_{n=1}^{\infty} \frac{(-1)^{n+1} x^{2n-1}}{(2n-1)!} \) & \( 0 < |x| < \pi \) \\
			\hline
			9 & \( \arcsin(x) \) & \( \displaystyle\sum_{n=0}^{\infty} \frac{(2n)! x^{2n+1}}{2^{2n} (n!)^2 (2n+1)} \) & \( |x| \leq 1 \) \\
			\hline
			10 & \( \arctan(x) \) & \( \displaystyle\sum_{n=1}^{\infty} \frac{(-1)^{n+1} x^{2n-1}}{(2n-1)} \) & \( |x| \leq 1 \) \\
			\hline
			11 & \( \sinh(x) \) & \( \displaystyle\sum_{n=0}^{\infty} \frac{x^{2n+1}}{(2n+1)!} \) & \( x \in \mathbb{C} \) \\
			\hline
			12 & \( \cosh(x) \) & \( \displaystyle\sum_{n=0}^{\infty} \frac{x^{2n}}{(2n)!} \) & \( x \in \mathbb{C} \) \\
			\hline
			13 & \( \tanh(x) \) & \( \displaystyle\sum_{n=1}^{\infty} \frac{x^{2n-1}}{(2n-1)} \) & \( |x| < \frac{\pi}{2} \) \\
			\hline
			14 & \( \coth(x) \) & \( \displaystyle\sum_{n=1}^{\infty} \frac{x^{2n-1}}{(2n-1)} \) & \( 0 < |x| < \pi \) \\
			\hline
			15 & \( \text{sech}(x) \) & \( \displaystyle\sum_{n=0}^{\infty} \frac{x^{2n}}{(2n)!} \) & \( |x| < \frac{\pi}{2} \) \\
			\hline
			16 & \( \text{csch}(x) \) & \( \displaystyle\sum_{n=1}^{\infty} \frac{x^{2n-1}}{(2n-1)!} \) & \( 0 < |x| < \pi \) \\
			\hline
			17 & \( \text{arsinh}(x) \) & \( \displaystyle\sum_{n=0}^{\infty} \frac{(2n)! x^{2n+1}}{2^{2n} (n!)^2 (2n+1)} \) & \( |x| \leq 1 \) \\
			\hline
			18 & \( \text{artanh}(x) \) & \( \displaystyle\sum_{n=1}^{\infty} \frac{x^{2n-1}}{(2n-1)} \) & \( |x| < 1 \) \\
			\hline
			19 & \( \text{arsinh}(x) \) & \( \displaystyle\sum_{n=0}^{\infty} \frac{(2n)! x^{2n+1}}{2^{2n} (n!)^2 (2n+1)} \) & \( |x| \leq 1 \) \\
			\hline
			20 & \( \sin(x) \) & \( \displaystyle\sum_{n=0}^{\infty} \frac{(-1)^n x^{2n+1}}{(2n+1)!} \) & \( x \in \mathbb{C} \) \\
			\hline
		\end{tabular}
	\end{table}
	
	\newpage	
	\section{Другий додаток}\label{sec:appendix2}
	
	\begin{table}[h!]
		\centering
		\caption{Варіанти індивідуальних завдань для створення таблиці}
		\label{tab:individual_tasks}
		\begin{tabular}{|c|c|c|c|}
			\hline
			№ варіанта & Логічний вираз & № варіанта & Логічний вираз \\
			\hline
			\hline
			1 & \( \lnot (A \land \lnot B) \lor B \) & 11 & \( \lnot (A \lor \lnot B) \lor \lnot B \land A \) \\
			\hline
			2 & \( A \land \lnot B \land \lnot (A \lor B) \) & 12 & \( (A \land B) \lor (\lnot A \land \lnot B) \) \\
			\hline
			3 & \( \lnot B \lor (A \land \lnot B) \) & 13 & \( \lnot A \lor B \lor (\lnot A \land B) \) \\
			\hline
			4 & \( \lnot A \land B \lor \lnot A \) & 14 & \( A \land \lnot B \land \lnot (A \lor B) \) \\
			\hline
			5 & \( \lnot (A \land B) \lor (B \lor \lnot A) \) & 15 & \( \lnot B \lor (A \land \lnot B) \) \\
			\hline
			6 & \( \lnot B \land (\lnot A \lor B) \) & 16 & \( \lnot (A \land B) \lor A \land \lnot B \) \\
			\hline
			7 & \( A \lor \lnot (B \lor A) \land A \) & 17 & \( A \land (\lnot B \lor B) \) \\
			\hline
			8 & \( \lnot B \land (A \lor B) \) & 18 & \( A \lor \lnot B \land \lnot A \) \\
			\hline
			9 & \( \lnot (A \lor \lnot B) \lor \lnot B \land A \) & 19 & \( \lnot (A \lor \lnot B) \land B \) \\
			\hline
			10 & \( (A \land B) \lor (\lnot A \land \lnot B) \) & 20 & \( A \land B \lor \lnot A \) \\
			\hline
		\end{tabular}
	\end{table}
	
\end{document}
